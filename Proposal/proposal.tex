\documentclass{article}

% if you need to pass options to natbib, use, e.g.:
%     \PassOptionsToPackage{numbers, compress}{natbib}
% before loading neurips_2018

% ready for submission
% \usepackage{neurips_2018}

% to compile a preprint version, e.g., for submission to arXiv, add add the
% [preprint] option:
%     \usepackage[preprint]{neurips_2018}

% to compile a camera-ready version, add the [final] option, e.g.:
     \usepackage[final]{proposal}

% to avoid loading the natbib package, add option nonatbib:
%     \usepackage[nonatbib]{neurips_2018}

\usepackage[utf8]{inputenc} % allow utf-8 input
\usepackage[T1]{fontenc}    % use 8-bit T1 fonts
\usepackage{hyperref}       % hyperlinks
\usepackage{url}            % simple URL typesetting
\usepackage{booktabs}       % professional-quality tables
\usepackage{amsfonts}       % blackboard math symbols
\usepackage{nicefrac}       % compact symbols for 1/2, etc.
\usepackage{microtype}      % microtypography
\usepackage{fontspec, xunicode, xltxtra} 
\setmainfont{Microsoft YaHei} 
\usepackage{ctex}


\title{《人工神经网络》大作业开题报告}

% The \author macro works with any number of authors. There are two commands
% used to separate the names and addresses of multiple authors: \And and \AND.
%
% Using \And between authors leaves it to LaTeX to determine where to break the
% lines. Using \AND forces a line break at that point. So, if LaTeX puts 3 of 4
% authors names on the first line, and the last on the second line, try using
% \AND instead of \And before the third author name.

\author{
  陶天骅\\
  计算机科学与技术系 \\
  清华大学 \\
  \texttt{tth17@mails.tsinghua.edu.cn} \\
  %% examples of more authors
  \AND
  杨雅儒\\
  计算机科学与技术系 \\
  清华大学 \\
  \texttt{yangyr17@mails.tinghua.edu.cn} \\
  %% \And
  %% Coauthor \\
  %% Affiliation \\
  %% Address \\
  %% \texttt{email} \\
}

\begin{document}
% \nipsfinalcopy is no longer used

\maketitle




\section{任务定义}

本课题希望构建一个神经网络以及一些简单的界面,可以根据用户提供的一些特征的比例,自动生成一张风景图片。程序界面示意图如下。

用户通过滑动滑条,确定例如树木、山、水、天空等要素在图片中的占比,并可以设定希望使用的主题颜色,程序便生成一张符合以上要求的风景图片。用户可以设置随机种子来获得不同的图片。

用数学语言形式化程序的任务即为:

\section{数据集}

用于训练的数据集可以是从各大图片社交平台(如 Pinterest 、Flicker)上下载获得的风景图片。

有一些现存的数据集可以使用,但是需要剔除一些内容,包括MIT的Computational Visual Cognition Laboratory,Github上的ml5-data-and-models,MIT Computer Science and Artificial Intelligence Laboratory提供的places数据集等。

目前估计图片的大小为256 x 256,数量在3000以上。

\section{挑战和基线}

\subsection{挑战}
\begin{itemize}
	\item 确定使用哪些特征作为输入标签。
	\item 将特征和主题颜色向量化。
	\item 考虑到算力有限,可能无法生成分辨率较高的图片。
	\item 融合不同的神经网络架构,打造一个本程序专用的神经网络。
\end{itemize}

\subsection{基线}

\begin{itemize}

\item 基线1.
Nvidia的SPADE项目构建了一个程序,可以根据景物轮廓绘出风景图像,与本项目有类似之处,但是它使用了不同的原理。Nvidia的工作构造了一个称为GauGAN的网络,在COCO-Stuff, Cityscapes 和 ADE20K 数据集上学习了图像语义分割,再由特定类别的图像语义生成图像。但是考虑到本项目不会涉及到非常多的语义分割的部分,且SPADE项目的规模要大的多,因而参考意义有限。
\item 基线2.

\end{itemize}

\section{研究计划}

\begin{enumerate}
	\item 收集训练数据
	计划使用手动(打包下载)或者自动(爬虫)的方法,在一些图片平台上获取与风景相关的图片数据。
	\item 清洗数据
	对各种图片,删去一些不相关的内容,调整为统一大小,统一格式。
	\item 构建基于CNN、GAN、AutoEncoder的神经网络
	\item 构建GUI
	\item 训练
	\item 测试
\end{enumerate}

\section{可行性}
本项目使用到的主要方法为一些十分成熟的神经网络架构,主要难点在于将他们进行融合,并设计本项目专用的网络,因为既能保证可行性,又有创新点。在计算规模上,我们尽量将图片的大小控制在可以在单个GPU上训练的规模。
\end{document}
