\documentclass{article}

% if you need to pass options to natbib, use, e.g.:
%     \PassOptionsToPackage{numbers, compress}{natbib}
% before loading neurips_2018

% ready for submission
% \usepackage{neurips_2018}

% to compile a preprint version, e.g., for submission to arXiv, add add the
% [preprint] option:
%     \usepackage[preprint]{neurips_2018}

% to compile a camera-ready version, add the [final] option, e.g.:
     \usepackage[final]{nips_2018}

% to avoid loading the natbib package, add option nonatbib:
%     \usepackage[nonatbib]{neurips_2018}

\usepackage[utf8]{inputenc} % allow utf-8 input
\usepackage[T1]{fontenc}    % use 8-bit T1 fonts
\usepackage{hyperref}       % hyperlinks
\usepackage{url}            % simple URL typesetting
\usepackage{booktabs}       % professional-quality tables
\usepackage{amsfonts}       % blackboard math symbols
\usepackage{nicefrac}       % compact symbols for 1/2, etc.
\usepackage{microtype}      % microtypography
\usepackage{graphicx}
\usepackage{CJKutf8}


\title{《人工神经网络》大作业报告}
% 标题可根据自己的project进行替换

% The \author macro works with any number of authors. There are two commands
% used to separate the names and addresses of multiple authors: \And and \AND.
%
% Using \And between authors leaves it to LaTeX to determine where to break the
% lines. Using \AND forces a line break at that point. So, if LaTeX puts 3 of 4
% authors names on the first line, and the last on the second line, try using
% \AND instead of \And before the third author name.

\author{%
  张三\thanks{可利用脚注提供作者的更多信息} \\
  2016000000 \\
  计算机系 \\
  \texttt{zhangsan@tsinghua.edu.cn} \\
  %% examples of more authors
  \And
  李四\\
  2015111111 \\
  计算机系 \\
  \texttt{lisi@outlook.com} \\
  %% \And
  %% Coauthor \\
  %% Affiliation \\
  %% Address \\
  %% \texttt{email} \\
}

\begin{document}
% \nipsfinalcopy is no longer used
\begin{CJK*}{UTF8}{gbsn}
\maketitle

\begin{abstract}
    摘要不超过300字。
\end{abstract}


\section{引言}

本部分介绍问题背景、解决问题的动机、遇到的挑战,并概述解决问题的方案。

\section{相关工作}

本部分综述与你的大作业相关的文献。

\section{方法}

本节描述大作业中使用的方法,包含问题的形式化、模型整体架构及模型每一部分的详细叙述。注意合理使用公式、图和表格。

\section{实验}

本节至少需要包含:数据集的详细说明、模型的参数设置、baseline模型说明、实验结果以及对结果的分析。实验分析部分需要同时提
供量化的数值分析(可以通过图、表格来呈现)和一些具体的结果示例。

\textbf{注意}: 可以参考开题报告中给出的图表示例来描述你的实验结果。


\section{结论}

总结本次大作业的贡献,说明组内的分工情况。


\section*{参考文献}

\small

[1] Alexander, J.A.\ \& Mozer, M.C.\ (1995) Template-based algorithms for
connectionist rule extraction. In G.\ Tesauro, D.S.\ Touretzky and T.K.\ Leen
(eds.), {\it Advances in Neural Information Processing Systems 7},
pp.\ 609--616. Cambridge, MA: MIT Press.

[2] Bower, J.M.\ \& Beeman, D.\ (1995) {\it The Book of GENESIS: Exploring
  Realistic Neural Models with the GEneral NEural SImulation System.}  New York:
TELOS/Springer--Verlag.

[3] Hasselmo, M.E., Schnell, E.\ \& Barkai, E.\ (1995) Dynamics of learning and
recall at excitatory recurrent synapses and cholinergic modulation in rat
hippocampal region CA3. {\it Journal of Neuroscience} {\bf 15}(7):5249-5262.

\end{CJK*}
\end{document}


